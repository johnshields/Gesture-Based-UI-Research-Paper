\documentclass{article}

%%%%%%% PACKAGES %%%%%%%%
\usepackage[utf8]{inputenc}
\usepackage[margin=2cm]{geometry}
\usepackage{blindtext}
\usepackage{setspace}
\usepackage{graphicx}
\usepackage{notoccite} %citation number ordering
\usepackage{lscape} %landscape table
\usepackage{caption} %add a newline in the table caption
\usepackage{float}
\usepackage{color}
\usepackage[dvipsnames]{xcolor}
\definecolor{ultramarine}{HTML}{2f3973}
\definecolor{hexablack}{HTML}{000000}
\usepackage[colorlinks = true,
linkcolor = hexablack,
urlcolor  = ultramarine,
citecolor = ultramarine,
anchorcolor = hexablack]{hyperref}

\title{\huge{\textbf{Gesture Based UI Development}} \\
\LARGE{Research Paper}}
\author{John Shields}
\date{February 2021}

\begin{document}
\pagenumbering{roman} % Start roman numbering
\clearpage\maketitle
\thispagestyle{empty}
\begin{center}
    \begin{figure}[h]
        \centering
        \includegraphics[width=15cm]{pics/logo-gmit.png}
        %\caption{Your caption here}
        \label{fig:logo}
    \end{figure}
    \large{	BSc (Hons) in Software Development \\
    Lecturer: Damien Costello}
\end{center}
\newpage
\setcounter{page}{1}
\tableofcontents
\listoffigures

\newpage
\pagenumbering{arabic} % Start roman numbering

%%% START OF CONTENT %%%%
\section{Description}
This Research Paper is based on Gesture Based User Interface Experience that focuses on Accessibility, Evolution, and Challenges. The purpose of this paper is to research the User Interface as it moves from purely physical (mouse, keyboard, touch screen) to include intuitive interaction through gestures.

\section{Introduction}
Gesture Based User Interface Experience can vary in many ways. Computers are all around us. It can be said that almost every household has at least one computer, where it be a PC or a Laptop, there is sure to be one in most homes today. Obviously, computers are not the only devices in homes. Nowadays, Phones are extremely popular and heavily used. They might even be used more than computers. Having said that, a phone is basically a computer that you can fit in your pocket. Gesture Based UI is heavily used in these devices. This can be as simple as sliding your hand to the left and right to go back and forth between pages for a computer. Gestured Based UI only came into play with phones when the touch screen was introduced. Computer too can have touch screens, but it is optional. Most personal phones likely have touch screens. Today's phones rely on the touch screen. They only have a few physical buttons. These are mainly for locking/unlocking phones and for adjusting the volume level. Gesture Based UI is not just for computers and phones. It is also for something as simple as a watch or a car radio, making them smarter. Gestured Based UI makes every device smarter.

\section{User Experience Evolution}
% Originally with just a keyboard and screen, computers have evolved to include
% a mouse, touch screen, voice control, virtual augmented spaces and gesture
% recognition. What is the User Experience and what are the drivers in the
% evolution process? Why are user’s never happy with the current iteration of
% interaction with computing systems? What are the challenges faced by each
% generation as new interactions are marketed to users from both acceptance and
% implementation perspectives?

\subsection{Early Days of User Experience}
User Experience is constantly Evolving. The first general-purpose computer was the monolithic ENIAC machine. The machine's construction began on the 13th of May, 1943, and was finished on the 2nd of October, 1955. The User Experience on the ENIAC was a bit of a headache and relied on many experts to work the machine. ENIAC was designed to be capable of being reprogrammed to solve a large number of numerical problems. The machine's programming consisted of setting switches and connecting wires according to specific instructions, which were first worked out on paper, which took weeks. This machine also used a punch-card reader as an input and output device. This is obviously not very user-friendly, but for the time, it was extraordinary. Comparing this machine to any computer or smart device today shows how much these technologies have evolved.
\cite{ref1}

\begin{figure}[h!]
    \caption{ENIAC Machine}
    \label{image:ENIAC}
    \centering
    \includegraphics[width=0.6\textwidth]{pics/eniac.jpg}
\end{figure}
\newpage
\subsection{The Computer Keyboard}
One of the key parts of a computer that is still highly used today is the keyboard. Typewriters inspire keyboards. The typewriter has been around since the late 1800s and is an antique that is still used today. The QWERTY keyboard layout, which came from the Sholes and Glidden typewriter, also came from this time. QWERTY's design has the keys to be spread out to avoid jamming those old typewriters. The QWERTY layout has stood the test of time in the English speaking world. The Binac computer (1948) used an electromechanically controlled typewriter to input data directly onto magnetic tape and print results. This was the first integration of a keyboard like system for a computer and began the computer keyboard's origins. Since this origins, the keyboard is used in most electronic devices that require typing. 
\cite{ref3}

\begin{figure}[h!]
    \caption{The Binac Computer}
    \label{image:BINAC}
    \centering
    \includegraphics[width=0.45\textwidth]{pics/binac.jpg}
\end{figure}

\subsection{The Computer Screen}
What is a UI without a screen? A screen is so essential now and is probably the most important part of a computer. The ZUSE 3 (Z3) Computer (1941) was the first working programmable and fully automatic digital computer.  The Z3 was constructed with hundreds of relays, second-hand sheet metal, and mechanical pins. The Z3 had an output display that showed results on a light stripe, including the location of decimal commas. At the same time, computers from this time provided hard-copy printouts. Computers like the Z3 were dominated by a digital display that consisted of rows of blinking lights that flashed when the computer processed particular instructions or accessing memory locations. 
\cite{ref4} \cite{ref5}

\begin{figure}[H]
    \caption{The ZUSE 3 Computer}
    \label{image:ZUSE3}
    \centering
    \includegraphics[width=0.5\textwidth]{pics/z3.jpg}
\end{figure}

Since the Z3, computers such as the SWAC Console and the Ferranti Mark 1 Star (1951) used CRT Display (Cathode-ray tube) to show CRT-based memory contents. In the 1960s, CRT was used as virtual paper in a virtual teletype on computers such as the Uniscope 300 (1964) and the ADM-3 (1975). These display terminals became very prominent for the UI in computers right into the mid-1970s. At the time, these computers were quite expensive, so they were not hugely accessible. In the mid-1970s, CCTV was used to create cheaper computers such as the Apple I (1976) and the SOL-20 (1976). These were the first computers that have factory video outputs. As a result, these computers became more accessible than their predecessors. With the Television (TV) becoming more and more popular, computers started to use the TV's technology. Companies like Apple, Commodore, and Radio shack used this technology to their advantage to create a better User Experience.

In the 1980s, computer screens became revolutionary with the use of CGA (Colour Graphics Adapter) and EGA (Enhanced Graphics Adapter) that were both introduced by IBM. These brought colour and higher resolution to computer screens. Followed by these displays, the Macintosh Monitors came, and they used bitmapped graphics. The Mac II's (1978) video standard was quite similar to the VGA (Video Graphics Array). Mac monitors continued to evolve and were always known for their sharpness and accurate colour representation.

Also, in the 1980s, LDC (Liquid Crystal Display) screens that originated from the 1960s became widely used on calculators and watches with monochrome displays—from the 1980s and on through the 1990s, LCD drastically improved. At this time, LCD created a market boom for laptops. LDC is continued to be supported and used today in electronic devices such as PCs, laptops, TVs, smartphones, smartwatches, cameras, Etc.
\cite{ref5}

\begin{figure}[!h]
    \caption{Modern Screens}
    \label{image:MODERNSCREENS}
    \centering
    \includegraphics[width=0.5\textwidth]{pics/modern_screens.png}
\end{figure}

\subsection{The Computer Mouse}
The Mouse is another essential part of a computer. Before the use of the mouse, data was entered by typing commands with the keyboard, as discussed earlier. The common mice that are used today are made out of plastic. The first computer mouse was designed as a wooden box with two metal wheels that make contact with the surface and only one key. Douglas Engelbart invented this mouse in 1964. This mouse was pretty limited in what it could do. It was only eight years later that Bill English created the Ball Mouse. The ball replaced the two metal wheels that Engelbart's mouse, which made it possible for this mouse to move in every direction. The Alto (1973) had a special input interface made by SRI. Alto's mouse had three buttons and enabled the first bitmapped and overlapping windows display. As a result, Alto's became very popular with its users. D. Venolia of Apple developed the first scroll-wheel in the late 1980s. It was not until the 1990s that mice started to resemble present-day mice. In 1999 Microsoft created a mouse known as the Intellimouse with an Optic LED design. This mouse paved the way for a new generation of optical mice.
\cite{ref6} \cite{ref7}

\begin{figure}[!ht]
    \caption{Computer Mice Throughout the Years}
    \label{image:mice}
    \centering
    \includegraphics[width=0.5\textwidth]{pics/mice.png}
\end{figure}

\newpage
\subsection{The Touch Screen}
Touch Screen technology has become vastly popular throughout the years. Touchscreens are everywhere. They can be on PCs, laptops, smartphones, smartwatches, cameras, and almost every other electronic device with a screen. The touchscreen originated in the 1960s. E.A. Johnson invented the first finger-driven touchscreen in 1965. This touchscreen mechanism is used in many smartphones today—what now is know as capacitive touch. An insulator, like glass, coated with a transparent conductor (such as Indium Tin Oxide) is used by the capacitive touchscreen panel. Usually, the 'conductive' part is a human finger or a touch pen. Johnson's initial touchscreen could only process one touch at a time, and what is describe today as 'multi-touch' was still years away. By 1971, a number of different machines with touchscreens were introduced. One of these machines was the Plato IV, which was the first touchscreen computer to be used in a classroom. This allowed students to touch the screen to select answers to questions. The PLATO IV employed infrared technology rather than capacitive or resistive. 
\cite{ref8}

\begin{figure}[!ht]
    \caption{The PLATO IV touchscreen terminal}
    \label{image:PLATOIV}
    \centering
    \includegraphics[width=0.5\textwidth]{pics/plato_iv.jpg}
\end{figure}

At the beginning of the 1980s, touchscreens began heavily commercialized. HP created a computer called the HP-150 in 1983. The HP-150 used MS-DOS and featured a 9inch Sony CRT surrounded by infrared (IR) emitters and detectors that sensed where the user's finger came down the screen. The HP-150 was not immediately embraced because it had usability issues. For example, poking at the screen would, in turn, block other IR rays that could tell the computer where the user's finger was pointing. 

IBM and BellSouth teamed up in 1993 to launch the Simon Personal Communicator (SPC). The SPC was one of the first cellphones that possessed a touchscreen. The phone featured paging capabilities, an e-mail and calendar application, an appointment schedule, an address book, a calculator, and a pen-based sketchpad. The touchscreen was resistive, which required a stylus to navigate through menus and to input data.
\cite{ref8}

\begin{figure}[!ht]
    \caption{IBM's Simon Personal Communicator}
    \label{image:SPC}
    \centering
    \includegraphics[width=0.4\textwidth]{pics/IBM_Simon.jpeg}
\end{figure}

In the 2000s, touchscreen technology flourished. There were advancements during this time that helped bring multi-touch and gesture-based technology to the masses. The 2000s were also the era when touchscreens became the favorite tool for design collaboration. The first human-controlled multi-touch device was invented at the University of Toronto in 1982 by Nimish Mehta. It featured a frosted glass panel in front of a camera, which detected action when it identified ‘black spots’ showing up onscreen.
As the 2000s approached, companies poured more resources into integrating touchscreen technology into their daily processes. With the arrival of the PortfolioWall 3D animators and designers were primarily targeted. PortfolioWall was a large-format touchscreen meant to be a dynamic version of the boards that design studios use to track projects.  The PortfolioWall had a gesture-based interface that was simple and easy-to-use. It allowed users to inspect and maneuver images, animations, and 3D files. 
\cite{ref8}

\begin{figure}[!ht]
    \caption{PortfolioWall}
    \label{image:PFW}
    \centering
    \includegraphics[width=0.5\textwidth]{pics/pf_wall.jpg}
\end{figure}

Sony unveiled a flat input surface named SmartSkin in 2002 that could simultaneously identify multiple hand configurations and touch-points by measuring the distance between the hand and the surface using capacitive sensing and a mesh-shaped antenna.
The project's ultimate aim was to turn surfaces seen every day, such as an ordinary table or a wall, into an immersive one using a nearby PC.

In 2007 Microsoft introduced the Surface at the All Things D conference. Although many of its design concepts were not new, it very effectively illustrated the real-world use case for touchscreens embedded into something the size of a coffee table. Microsoft designed the Surface to be used by their commercial customers to give users a sample of the hardware. Firms such as AT\&T used the Surface to display the latest handsets to buyers visiting their brick and mortar retail locations.

These technologies cannot be understated, as each had a monumental impact on the electronic devices we use today. Everything from PCs, laptops, smartphones, smartwatches, cameras can connect to the numerous innovations, discoveries, and patents in the history of touchscreen technology.
\cite{ref8}


\subsection{Voice Control}

\newpage
\subsection{Virtual and Augmented Spaces}

\subsection{Gesture Recognition}

\section{Gestures as a Communication Tool}
% How are gestures used in everyday life for communication with others? They
% are a universal tool albeit without a unilateral meaning and interpretation.
% How are gestures defined and become accepted to represent various aspects of
% communication?

\subsection{How are gestures used in everyday life for communication with others?}
\subsection{How are gestures defined and become accepted to represent various aspects of
communication?}

\section{Challenges for the Design of Applications}
% Incorporating gestures is an important part of the design phase. Using the
% wrong gestures will leave users confused and frustrated as they learn a new
% system. The functionality of the system has to be appropriately mapped to
% the gesture set available in such a way to reduce the learning curve and the
% resistance gradient of the user

\section{Challenges for Implementation}
% Deciding that a particular gesture should carry out a particular function is one
% thing, actually tracking that gesture and deciding when it has been made is a
% different challenge. Looking at some systems and the gestures they use, what
% are the challenges in implementation that had to be overcome for those systems?
% How were those challenges met (if they were met)?

\section{Conclusions}
% Conclusions are what you have learned from your research. This is your reflection on the current state of the art and the possible future directions of gesture-based user interfaces.


%%% END OF CONTENT %%%%

 \newpage
 \begin{thebibliography}{00}
    
\bibitem{ref1} History of Computers
\newline
URL: \url{https://homepage.cs.uri.edu/faculty/wolfe/book/Readings/Reading03.htm}

\bibitem{ref2} Suzanne Deffre - EDN - Construction begins on ENIAC
\newline
URL: \url{https://www.edn.com/construction-begins-on-eniac-may-31-1943/}

\bibitem{ref3} Mary Bellis - The History of the Computer Keyboard
\newline
URL: \url{https://www.thoughtco.com/history-of-the-computer-keyboard-1991402}

\bibitem{ref4} History Computer Konrad Zuse
\newline
URL: \url{https://history-computer.com/konrad-zuse/}

\bibitem{ref5} Benji Edwards - PC world - A brief history of computer displays
\newline
URL: \url{https://www.pcworld.idg.com.au/slideshow/366677/brief-history-computer-displays/}

\bibitem{ref6} Jessica Z - Sutori - HISTORY OF COMPUTER MOUSE
\newline
URL: \url{https://www.sutori.com/story/history-of-computer-mouse--2yUFPn6vNQBstaaz2x4FTdsy}

\bibitem{ref7} Bill Buxton - Some Milestones in Computer Input Devices
\newline
URL: \url{https://www.billbuxton.com/inputTimeline.html}

\bibitem{ref8} Florence Ion - From touch displays to the Surface
\newline
URL: \url{https://tinyurl.com/ycpwsg8m}

\end{thebibliography}

%%% CONTENT HERE END %%%%
\end{document}
\newpage
\setstretch{1}  %reduce bibliography line spacing
\printbibliography
\end{document}