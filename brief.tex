\documentclass{article}
\begin{document}

\textbf{Project Description}

\section{User Experience Evolution}
Originally with just a keyboard and screen, computers have evolved to
include a mouse, touch screen, voice control, virtual & augmented spaces and gesture recognition.
What is the User Experience and what are the drivers in the evolution process? Why are user’s
never happy with the current iteration of interaction with computing systems? What are the
challenges faced by each generation as new interactions are marketed to users from both
acceptance and implementation perspectives?

\section{Gestures as a communication tool}
How are gestures used in everyday life for communication with
others? They are a universal tool albeit without a unilateral meaning and interpretation. How are
gestures defined and become accepted to represent various aspects of communication?

\section{Challenges for design of applications}
Incorporating gestures is an important part of the design
phase. Using the wrong gestures will leave users confused and frustrated as they learn a new
system. The functionality of the system has to be appropriately mapped to the gesture set available in such a way to reduce the learning curve and the resistance gradient of the user


\section{Challenges for implementation}
Deciding that a particular gesture should carry out a particular
function is one thing, actually tracking that gesture and deciding when it has been made is a
different challenge. Looking at some systems and the gestures they use, what are the challenges in
implementation that had to be overcome for those systems? How were those challenges met (if
they were met)?

\section{Conclusions}
Conclusions are what you have learned from your research. This is your reflection on
the current state of the art and the possible future directions of gesture-based user interfaces.

\end{document}